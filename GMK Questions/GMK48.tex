\documentclass{article}
\usepackage{amsfonts, amsmath, amssymb}
\usepackage[normalem]{ulem}
\usepackage{enumerate, mdwlist}

\newcounter{problem}
\newcounter{solution}

\newcommand\Problem{%
  \stepcounter{problem}%
  \textbf{\theproblem.}~%
  \setcounter{solution}{0}%
}

\newcommand\TheSolution{%
  \textbf{Solution:} %
}

\newcommand\Proof{%
	\textbf{Proof:} %
}

\parindent 0in
\parskip 1em

\begin{document}
Name: Jeremy Florence\\
Course: Math 460\\
Assignment: GMK \#48 Presentation\\
Due: 4/12/17\\

48.	Let $G$ be a group with identity $e$ and finite order $n$. Which of the following conditions is sufficient for $G$ to be abelian?

\begin{enumerate}[I.]
\item $n=6$
\suspend{enumerate}

\textbf{Counterexample:} Let $G=S_3=\{(),(1,2),(2,3),(1,3),(1,2,3),(1,3,2)\}$, the symmetric group on a set of three elements. It is clear that $|S_3|=6$. Now consider the elements $(1,2)$ and $(2,3)$ in $S_3$. Notice that $(1,2) \cdot (2,3)=(1,2,3)$ but $(2,3) \cdot (1,2)=(1,3,2).$ Hence $(1,2) \cdot (2,3) \neq (2,3) \cdot (1,2)$, so $S_3$ is not abelian. Thus there exists a group $G$ with order n=6 where $G$ is not abelian. Therefore $|G|=6$ is not a sufficient condition for $G$ to be abelian

\resume{enumerate}[{[I.]}]
\item $n=15$
\suspend{enumerate}

\textbf{Definition:} Let $G$ be a group and let $p$ be a prime. 
\begin{enumerate}[(1)]
\item A group of order $p^k$ for some $k \geq 0$ is a called a \emph{p-group}. Subgroups of $G$ which are $p$-groups are called \emph{p-subgroups}.
\item If $G$ is a group of order $p^km$, where $p\nmid m,$ then a subgroup of order $p^k$ is called a \emph{Sylow $p$-subgroup} of $G$.
\item The number of Sylow $p$-subgroups of G will be denoted by $n_p$.
\end{enumerate}

\textbf{Sylow's Theorem:} Let $G$ be a group of order $p^km$, where $p$ is a prime not dividing $m$. Then the following are true:
\begin{enumerate}[(1)]
\item Sylow $p$-subgroups of G exist.
\item If $P$ is a Sylow $p$-subgroup of $G$ and $Q$ is any $p$-subgroup of $G$, then there exists $g \in G$ such that $Q$ is a subgroup of $gPg^{-1}$, i.e., $Q$ is contained in some conjugate of $P$. In particular, any two Sylow $p$-subgroups of $G$ are conjugate in $G$.
\item $n_p \equiv 1 $(mod $ p)$, and $n_p$ divides $m$
\end{enumerate}

\Proof Let $G$ be a group with order $n=15$. Since $n=15=3 \cdot 5$, and both 3 and 5 are prime, we know that Sylow 3-subgroups of $G$ and Sylow-5 subgroups of $G$ exist by part (1) of Sylow's Theorem. Now by part (3) of Sylow's Theorem, we know that $n_3 \equiv 1 $(mod $ 3)$ and $n_3 \mid 5$. Thus as 5 is prime it must follow that $n_3=1$. Following the same reasoning, we can see that $n_5=1$. Let $P$ be the Sylow 3-subgroup in G and let Q be the Sylow 5-subgroup in G. Recall that $P \cap Q$ is a subgroup of $Q$ and $P$. Thus by Lagrange's Theorem, we know that $|P \cap Q|$ divides 3 and 5, so it follows that $|P \cap Q|=1$. That is, $P \cap Q=\{e\}$. Now consider the elements $a \in P$ and $b \in Q$. Now notice that $ab \notin P$ and $ab \notin Q$, and furthermore that $|\langle ab\rangle|$ must divide 15 by Lagrange's Theorem. Hence $|\langle ab \rangle|$ must be 1, 3, 5, or 15. Obviously $|\langle ab \rangle| \neq 1$ because this would imply $ab=e$, and further that $ab \in P \cap Q$, a contradiction. Also, $|\langle ab \rangle|$ cannot be 3, as this would imply that $\langle ab \rangle$ is a Sylow 3-subgroup of G, and moreover as $n_3=1$, $\langle ab \rangle = P$, which contradicts $ab \notin P$. For the same reasoning, we can conclude that $|\langle ab \rangle| \neq 5$. Therefore $|\langle ab \rangle|=15$, which means that $ab$ generates the entire group $G$. Hence $G$ is cyclic and therefore also abelian.  


\resume{enumerate}[{[I.]}]
\item $n$ is a prime number
\suspend{enumerate}

\Proof Let $G$ be a group with prime order p. Then $|G|>1$, so let $g \in G$ such that $g \neq e$. Then it follows that $|\langle g \rangle|>1$. Additionally, we know by Lagrange's Theorem that $|\langle g \rangle|$ divides $p$. Hence either $|\langle g \rangle|=1$ or $|\langle g \rangle|=p$. Thus as we already know $|\langle g \rangle|>1$, it must follow that $|\langle g \rangle|=p$. Therefore $\langle g \rangle=G$, and $g$ generates $G$. Hence $G$ is cyclic and therefore abelian.

\resume{enumerate}[{[I.]}]
\item $(ab)^2=a^2b^2$ for all $a,b \in G$

\end{enumerate}

\Proof Let $G$ be a group such that $(ab)^2=a^2b^2$ for all $a,b \in G$. Let $a,b \in G$. Then it follows that:
\[
	\begin{split}
	(ab)^2&=a^2b^2\\
	(ab)(ab)b^{-1}&=a^2b^2b^{-1}\\
	aba&=a^2b\\
	a^{-1}aba&=a^{-1}a^2b\\
	ba&=ab.
	\end{split}
\]

Therefore as $ab=ba$, we can conclude that $G$ is abelian.\\

\textbf{Answer: D (II, III, and IV only)}



\end{document}