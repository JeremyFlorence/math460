\documentclass{article}
\usepackage{amsfonts, amsmath}
\usepackage[normalem]{ulem}
\usepackage{blindtext}
\usepackage[shortlabels]{enumitem}

\newcounter{problem}
\newcounter{solution}

\newcommand\Problem{%
  \stepcounter{problem}%
  \textbf{\theproblem.}~%
  \setcounter{solution}{0}%
}

\newcommand\TheSolution{%
  \textbf{Solution:} %
}

\newcommand\Proof{%
	\textbf{Proof:} %
}

\parindent 0in
\parskip 1em

\begin{document}
Name: Jeremy Florence\\
Course: Math 460\\
Assignment: Weekly \#12\\
Due: 4/24/17\\

\textbf{12.1} To avoid paying a toll of \$2 on a direct road, Mr. Fred Cheaply first goes west 10 miles, then south 5 miles, then 30 miles west and finally 35 miles north. Fred’s old Hyundai gets 30 miles to the
gallon and he can still buy gas for \$1.659 per gallon.

\begin{enumerate}[(a)]
\item How far out of his way is Fred going to avoid paying the \$2 toll?\\

\TheSolution Observe that Fred is travelling north-west. Thus we must calculate the net distance traveled by Fred in the north and west directions:
		$$d_{north}=35-5=30$$
		$$d_{west}=10+30=40$$
Now using Pythagorean's Theorem, we can calculate the distance of the direct road:
		$$d_{direct}=\sqrt{30^2+40^2}=50$$
Therefore by avoiding the \$2 toll, Fred travels a total distance of $10+5+30+35=80$ miles. Hence Fred is going $80-50=30$ miles out of his way in order to avoid the \$2 toll.
\item On the basis of gas mileage his Hyundai gets and what he pays per gallon of gas, is Fred saving any money (and if so how much)?

\TheSolution If Fred had taken the direct road he would have used $50/30=5/3$ gallons of gas. Thus the total cost of using the direct road would be:
		$$cost_{direct}=(5/3) \cdot \$1.659+\$2=\$4.765.$$
Now by not taking the direct road, Fred uses $80/30=8/3$ gallons of gas. Thus the cost of not taking the direct road is $$cost_{indirect}=(8/3) \cdot \$1.659=\$4.424.$$

Therefore Fred saves $\$4.765-\$4.424=\$0.34$ by not taking the direct road.

\end{enumerate}

\textbf{12.2} Given that $a$, $b$ and $c$ are odd integers, prove that the equation
										$$ax^2+bx+c=0$$
cannot have a rational root (i.e. a root of the form $\frac{p}{q}$ where $p \in \mathbb{Z}$ and $q \in \mathbb{N}$.

\emph{Hint: Consider $b^2-4ac$ mod 8.}

\Proof Let $a,b$, and $c$ be odd integers so that $ax^2+bx+c=0$. Recall that a quadratic equation has rational roots if and only if there exists $d \in \mathbb{Z}$ so that $d^2=b^2-4ac$. Now notice that as $a,b$, and $c$ are odd, $b^2$ must be odd and $4ac$ must be even. Hence $b^2-4ac$ is odd, so $d$ must also be odd.

\textbf{Claim:} For any odd integer $n$, $n^2 \equiv 1$ (mod 8)

\textbf{Proof of Claim:} Let $n$ be an odd integer. We need to show that $8 \mid n^2-1$. Since $n$ is odd, there exists $k \in \mathbb{Z}$ so that $n=2k+1$. Thus
\[
	\begin{split}
	n^2-1&=(2k+1)^2-1\\
		&=4k^2+4k+1-1\\
		&=4(k^2+k).
	\end{split}
\]
Now notice that regardless of whether $k$ is even or odd, $k^2+k$ must be even. Hence there exists $j \in \mathbb{Z}$ so that $k^2+k=2j$, so $$n^2-1=4(k^2+k)=4(2j)=8j.$$
Therefore we can conclude that $8 \mid n^2-1$, so $n^2 \equiv 1$ (mod 8).\\

Now we know that $d^2 \equiv 1$ (mod 8), and also that $b^2 \equiv 1$ (mod 8). Next we will show that $-4ac \equiv 4$ (mod 8). Thus we will show that $8 \mid -4ac-4$. Notice that $ac$ is odd, so $ac+1$ must be even. That is, there exists some $k \in \mathbb{Z}$ so that $ac+1=2k$. Thus:
\[
	\begin{split}
	-4ac-4&=-4(ac+1)\\
		&=-4(2k)\\
		&=8 \cdot -k
	\end{split}
\]
so we can conclude that $-4ac \equiv 4$ (mod 8). Therefore $b^2+(-4ac) \equiv 1+4$ (mod 8), so $b^2-4ac \not \equiv 1$ (mod 8). Hence $b^2-4ac$ is not a square of some integer, so the equation $$ax^2+bx+c=0$$
cannot have a rational root.
\end{document}