\documentclass{article}
\usepackage{amsfonts, amsmath}
\usepackage[normalem]{ulem}
\usepackage{graphicx}
\usepackage{blindtext}

\newcounter{problem}
\newcounter{solution}

\newcommand\Problem{%
  \stepcounter{problem}%
  \textbf{\theproblem.}~%
  \setcounter{solution}{0}%
}

\newcommand\TheSolution{%
  \textbf{Solution:} %
}

\newcommand\Proof{%
	\textbf{Proof:} %
}

\parindent 0in
\parskip 1em


\begin{document}
Name: Jeremy Florence\\
Course: Math 460\\
Assignment: Weekly \#11\\
Due: 4/17/17\\

\textbf{11.1} In the diagram below, the square $ABCD$ has side equal to 7 cm. The square $EFGH$ is inscribed in the square $ABCD$ in such a way that $AE=BF=CG=DH=$ 3 cm.

\begin{figure}[htp]
\includegraphics[scale=0.50]{"Screenshot from 2017-04-11 12-53-08".png}
\end{figure}


(a) Find the area of square $EFGH$.\\

\TheSolution Notice that $AB=AE+EB$, $BC=BF+FC$, $CD=CG+GD$, $DA=DH+HA$. Thus as $AB=BC=CD=DA=7$ cm, and $AE=BF=CG=DH=3$ cm, we can see that $EB=FC=GD=HA=4$. Now by applying the Pythagorean Theorem we can compute the length of the side of $EFGH$:
\[
	\begin{split}
	(EF)^2 &= 3^2+4^2\\
	(EF)^2 &= 25\\
		EF &= 5 \text{ cm}
	\end{split}
\]

Therefore the area of $EFGH$ is $(5 \text{ cm})^2=25$ cm$^2$.\\

(b) Find the radius of the small circle (assumed to be inscribed in the triangle $EBF$).\\
(c) Assuming that the big circle is inscribed in the square $EFGH$, find the ratio of it's area to that of the smaller circle.\\

\TheSolution

\textbf{11.2} If the length of a major diagonal of a rectangular box is 1, prove that the total surface area of the box is at most 2.

\textbf{11.3} Prove that the surface area in the problem \textbf{11.2} is exactly 2 if and only if the rectangular box is a cube.
\end{document}